\documentclass[11pt,]{article}
\usepackage[]{mathpazo}
\usepackage{amssymb,amsmath}
\usepackage{ifxetex,ifluatex}
\usepackage{fixltx2e} % provides \textsubscript
\ifnum 0\ifxetex 1\fi\ifluatex 1\fi=0 % if pdftex
  \usepackage[T1]{fontenc}
  \usepackage[utf8]{inputenc}
\else % if luatex or xelatex
  \ifxetex
    \usepackage{mathspec}
  \else
    \usepackage{fontspec}
  \fi
  \defaultfontfeatures{Ligatures=TeX,Scale=MatchLowercase}
\fi
% use upquote if available, for straight quotes in verbatim environments
\IfFileExists{upquote.sty}{\usepackage{upquote}}{}
% use microtype if available
\IfFileExists{microtype.sty}{%
\usepackage{microtype}
\UseMicrotypeSet[protrusion]{basicmath} % disable protrusion for tt fonts
}{}
\usepackage[margin=1in]{geometry}
\usepackage{hyperref}
\hypersetup{unicode=true,
            pdftitle={MPs Parliament Attendance Data collation},
            pdfborder={0 0 0},
            breaklinks=true}
\urlstyle{same}  % don't use monospace font for urls
\usepackage{natbib}
\bibliographystyle{apsr}
\usepackage{color}
\usepackage{fancyvrb}
\newcommand{\VerbBar}{|}
\newcommand{\VERB}{\Verb[commandchars=\\\{\}]}
\DefineVerbatimEnvironment{Highlighting}{Verbatim}{commandchars=\\\{\}}
% Add ',fontsize=\small' for more characters per line
\usepackage{framed}
\definecolor{shadecolor}{RGB}{248,248,248}
\newenvironment{Shaded}{\begin{snugshade}}{\end{snugshade}}
\newcommand{\KeywordTok}[1]{\textcolor[rgb]{0.13,0.29,0.53}{\textbf{#1}}}
\newcommand{\DataTypeTok}[1]{\textcolor[rgb]{0.13,0.29,0.53}{#1}}
\newcommand{\DecValTok}[1]{\textcolor[rgb]{0.00,0.00,0.81}{#1}}
\newcommand{\BaseNTok}[1]{\textcolor[rgb]{0.00,0.00,0.81}{#1}}
\newcommand{\FloatTok}[1]{\textcolor[rgb]{0.00,0.00,0.81}{#1}}
\newcommand{\ConstantTok}[1]{\textcolor[rgb]{0.00,0.00,0.00}{#1}}
\newcommand{\CharTok}[1]{\textcolor[rgb]{0.31,0.60,0.02}{#1}}
\newcommand{\SpecialCharTok}[1]{\textcolor[rgb]{0.00,0.00,0.00}{#1}}
\newcommand{\StringTok}[1]{\textcolor[rgb]{0.31,0.60,0.02}{#1}}
\newcommand{\VerbatimStringTok}[1]{\textcolor[rgb]{0.31,0.60,0.02}{#1}}
\newcommand{\SpecialStringTok}[1]{\textcolor[rgb]{0.31,0.60,0.02}{#1}}
\newcommand{\ImportTok}[1]{#1}
\newcommand{\CommentTok}[1]{\textcolor[rgb]{0.56,0.35,0.01}{\textit{#1}}}
\newcommand{\DocumentationTok}[1]{\textcolor[rgb]{0.56,0.35,0.01}{\textbf{\textit{#1}}}}
\newcommand{\AnnotationTok}[1]{\textcolor[rgb]{0.56,0.35,0.01}{\textbf{\textit{#1}}}}
\newcommand{\CommentVarTok}[1]{\textcolor[rgb]{0.56,0.35,0.01}{\textbf{\textit{#1}}}}
\newcommand{\OtherTok}[1]{\textcolor[rgb]{0.56,0.35,0.01}{#1}}
\newcommand{\FunctionTok}[1]{\textcolor[rgb]{0.00,0.00,0.00}{#1}}
\newcommand{\VariableTok}[1]{\textcolor[rgb]{0.00,0.00,0.00}{#1}}
\newcommand{\ControlFlowTok}[1]{\textcolor[rgb]{0.13,0.29,0.53}{\textbf{#1}}}
\newcommand{\OperatorTok}[1]{\textcolor[rgb]{0.81,0.36,0.00}{\textbf{#1}}}
\newcommand{\BuiltInTok}[1]{#1}
\newcommand{\ExtensionTok}[1]{#1}
\newcommand{\PreprocessorTok}[1]{\textcolor[rgb]{0.56,0.35,0.01}{\textit{#1}}}
\newcommand{\AttributeTok}[1]{\textcolor[rgb]{0.77,0.63,0.00}{#1}}
\newcommand{\RegionMarkerTok}[1]{#1}
\newcommand{\InformationTok}[1]{\textcolor[rgb]{0.56,0.35,0.01}{\textbf{\textit{#1}}}}
\newcommand{\WarningTok}[1]{\textcolor[rgb]{0.56,0.35,0.01}{\textbf{\textit{#1}}}}
\newcommand{\AlertTok}[1]{\textcolor[rgb]{0.94,0.16,0.16}{#1}}
\newcommand{\ErrorTok}[1]{\textcolor[rgb]{0.64,0.00,0.00}{\textbf{#1}}}
\newcommand{\NormalTok}[1]{#1}
\usepackage{graphicx,grffile}
\makeatletter
\def\maxwidth{\ifdim\Gin@nat@width>\linewidth\linewidth\else\Gin@nat@width\fi}
\def\maxheight{\ifdim\Gin@nat@height>\textheight\textheight\else\Gin@nat@height\fi}
\makeatother
% Scale images if necessary, so that they will not overflow the page
% margins by default, and it is still possible to overwrite the defaults
% using explicit options in \includegraphics[width, height, ...]{}
\setkeys{Gin}{width=\maxwidth,height=\maxheight,keepaspectratio}
\IfFileExists{parskip.sty}{%
\usepackage{parskip}
}{% else
\setlength{\parindent}{0pt}
\setlength{\parskip}{6pt plus 2pt minus 1pt}
}
\setlength{\emergencystretch}{3em}  % prevent overfull lines
\providecommand{\tightlist}{%
  \setlength{\itemsep}{0pt}\setlength{\parskip}{0pt}}
\setcounter{secnumdepth}{0}
% Redefines (sub)paragraphs to behave more like sections
\ifx\paragraph\undefined\else
\let\oldparagraph\paragraph
\renewcommand{\paragraph}[1]{\oldparagraph{#1}\mbox{}}
\fi
\ifx\subparagraph\undefined\else
\let\oldsubparagraph\subparagraph
\renewcommand{\subparagraph}[1]{\oldsubparagraph{#1}\mbox{}}
\fi

%%% Use protect on footnotes to avoid problems with footnotes in titles
\let\rmarkdownfootnote\footnote%
\def\footnote{\protect\rmarkdownfootnote}

%%% Change title format to be more compact
\usepackage{titling}

% Create subtitle command for use in maketitle
\newcommand{\subtitle}[1]{
  \posttitle{
    \begin{center}\large#1\end{center}
    }
}

\setlength{\droptitle}{-2em}
  \title{MPs Parliament Attendance Data collation}
  \pretitle{\vspace{\droptitle}\centering\huge}
  \posttitle{\par}
  \author{true}
  \preauthor{\centering\large\emph}
  \postauthor{\par}
  \predate{\centering\large\emph}
  \postdate{\par}
  \date{Y}


\begin{document}
\maketitle

\begin{Shaded}
\begin{Highlighting}[]
\KeywordTok{library}\NormalTok{(dplyr)}
\KeywordTok{library}\NormalTok{(magrittr)}
\end{Highlighting}
\end{Shaded}

\subsection{16th Loksabha Sessions}\label{th-loksabha-sessions}

Session dates and attendance of all members taken from the official
\href{http://loksabha.nic.in}{website}.

\begin{Shaded}
\begin{Highlighting}[]
\NormalTok{## return the number of weekdays between dates d1 and d2\}}
\NormalTok{num_weekdays <-}\StringTok{ }\ControlFlowTok{function}\NormalTok{(d1, d2)\{}
\NormalTok{    weekend  <-}\StringTok{ }\KeywordTok{c}\NormalTok{(}\StringTok{"Saturday"}\NormalTok{, }\StringTok{"Sunday"}\NormalTok{)}
\NormalTok{    all_days <-}\StringTok{ }\KeywordTok{seq}\NormalTok{(d1, d2, }\StringTok{"1 day"}\NormalTok{) }\OperatorTok\StringTok{ }\NormalTok{weekdays}
    \KeywordTok{sum}\NormalTok{(}\OperatorTok{!}\NormalTok{(all_days }\OperatorTok\StringTok{ }\NormalTok{weekend))}
\NormalTok{\}}

\NormalTok{num_weekdays_v <-}\KeywordTok{Vectorize}\NormalTok{(num_weekdays)}
\end{Highlighting}
\end{Shaded}

I copied the session dates into a csv file. Two sessions (IV and XIV
sessions) had a break. We calculate the number of weekdays in each
session.

\begin{Shaded}
\begin{Highlighting}[]
\NormalTok{session_dates <-}\StringTok{ }\NormalTok{here}\OperatorTok{::}\KeywordTok{here}\NormalTok{(}\StringTok{"data/16th"}\NormalTok{, }\StringTok{"loksabha_session_dates.csv"}\NormalTok{)}
\NormalTok{session_dates <-}\StringTok{ }\NormalTok{readr}\OperatorTok{::}\KeywordTok{read_csv}\NormalTok{(session_dates)}
\end{Highlighting}
\end{Shaded}

\begin{verbatim}
## Parsed with column specification:
## cols(
##   session = col_character(),
##   start = col_character(),
##   end = col_character()
## )
\end{verbatim}

\begin{Shaded}
\begin{Highlighting}[]
\NormalTok{session_dates}\OperatorTok{$}\NormalTok{session }\OperatorTok\StringTok{ }\NormalTok{as.roman }\OperatorTok\StringTok{ }\NormalTok{as.integer}

\NormalTok{session_dates}\OperatorTok{$}\NormalTok{start }\OperatorTok\StringTok{ }\NormalTok{lubridate}\OperatorTok{::}\KeywordTok{dmy}\NormalTok{(.)}
\NormalTok{session_dates}\OperatorTok{$}\NormalTok{end   }\OperatorTok\StringTok{ }\NormalTok{lubridate}\OperatorTok{::}\KeywordTok{dmy}\NormalTok{(.)}

\NormalTok{session_dates}\OperatorTok{$}\NormalTok{session_length <-}\StringTok{ }\KeywordTok{num_weekdays_v}\NormalTok{(session_dates}\OperatorTok{$}\NormalTok{start,}
\NormalTok{                                               session_dates}\OperatorTok{$}\NormalTok{end)}

\NormalTok{sessions_durations <-}\StringTok{ }\NormalTok{session_dates }\OperatorTok
\StringTok{                      }\KeywordTok{group_by}\NormalTok{(session) }\OperatorTok
\StringTok{                      }\KeywordTok{summarise_if}\NormalTok{(is.numeric, sum)}
\end{Highlighting}
\end{Shaded}

The attendance information is unfortunately made available in doc
format. There were 14 files (one for each session). I converted each doc
file to docx and used the docxtractr package to extract the tables. I
know I've to extract the 4th table from each document because I had
queried the table structure using the \(docx_describe_tbls\) from the
package.

\begin{Shaded}
\begin{Highlighting}[]
\NormalTok{infile     <-}\StringTok{ }\NormalTok{here}\OperatorTok{::}\KeywordTok{here}\NormalTok{(}\StringTok{"data/16th"}\NormalTok{, }\KeywordTok{paste0}\NormalTok{(}\DecValTok{1}\OperatorTok{:}\DecValTok{14}\NormalTok{, }\StringTok{".docx"}\NormalTok{))}
\NormalTok{attendance <-}\KeywordTok{lapply}\NormalTok{(infile, }\ControlFlowTok{function}\NormalTok{(fname)\{}
\NormalTok{                                docxtractr}\OperatorTok{::}\KeywordTok{read_docx}\NormalTok{(fname) }\OperatorTok
\StringTok{                                }\NormalTok{docxtractr}\OperatorTok{::}\KeywordTok{docx_extract_tbl}\NormalTok{(}\DecValTok{4}\NormalTok{)\})}
\end{Highlighting}
\end{Shaded}

Stacking together the information from all sessions, I get a usable data
frame after having cleaned up the column names and fixed column classes.

\begin{Shaded}
\begin{Highlighting}[]
\KeywordTok{names}\NormalTok{(attendance) <-}\StringTok{ }\DecValTok{1}\OperatorTok{:}\DecValTok{14}
\NormalTok{all_sessions <-}\StringTok{ }\KeywordTok{bind_rows}\NormalTok{(attendance, }\DataTypeTok{.id =} \StringTok{"session"}\NormalTok{)}

\NormalTok{all_sessions }\OperatorTok\StringTok{ }\NormalTok{janitor}\OperatorTok{::}\KeywordTok{clean_names}\NormalTok{()}
\NormalTok{all_sessions}\OperatorTok{$}\NormalTok{session }\OperatorTok\StringTok{ }\NormalTok{as.integer}
\NormalTok{all_sessions}\OperatorTok{$}\NormalTok{no_of_days_member_signed_the_register }\OperatorTok\StringTok{ }\NormalTok{as.integer}
\end{Highlighting}
\end{Shaded}

Some of the column names are really clunky, so lets fix that.

\begin{Shaded}
\begin{Highlighting}[]
\NormalTok{all_sessions }\OperatorTok\StringTok{ }\NormalTok{dplyr}\OperatorTok{::}\KeywordTok{rename}\NormalTok{(}\DataTypeTok{mp_name =}\NormalTok{ name_of_member,}
                               \DataTypeTok{days_attended =}
\NormalTok{                               no_of_days_member_signed_the_register)}
\end{Highlighting}
\end{Shaded}

Finally, append the length of the session to this data frame and add a
column for percentage attendance.

\begin{Shaded}
\begin{Highlighting}[]
\NormalTok{all_sessions }\OperatorTok\StringTok{ }\KeywordTok{left_join}\NormalTok{(sessions_durations)}
\end{Highlighting}
\end{Shaded}

\begin{verbatim}
## Joining, by = "session"
\end{verbatim}

\begin{Shaded}
\begin{Highlighting}[]
\NormalTok{all_sessions }\OperatorTok\StringTok{ }
\StringTok{    }\KeywordTok{mutate}\NormalTok{(}\DataTypeTok{percent_attendance =} \DecValTok{100} \OperatorTok{*}\StringTok{ }\NormalTok{days_attended }\OperatorTok{/}\NormalTok{session_length)}
\end{Highlighting}
\end{Shaded}

\subsection{Features}\label{features}

What we have is a bare bones data set that records the name of each MP,
their constituency and the number of days they attended the parliament
(or at least, signed the register). I will now add some features such as
gender, education etc. that I care about to this information. The source
for this information is PRS Legislative Research, specifically
\href{http://www.prsindia.org/mptrack/16loksabha/}{this}. While the data
is well-formatted and this file also includes percent attendance, I
chose not to use it as it was not clear how this percent has been
calculated i.e., what was the denominator in each case. It was also
becoming a tad annoying to extract the dates hidden in the notes.

\begin{Shaded}
\begin{Highlighting}[]
\NormalTok{infile <-}\StringTok{ }\NormalTok{here}\OperatorTok{::}\KeywordTok{here}\NormalTok{(}\StringTok{"data"}\NormalTok{, }\StringTok{"MPTrack-16.xls"}\NormalTok{)}

\NormalTok{mps_attributes <-}\StringTok{ }\NormalTok{readxl}\OperatorTok{::}\KeywordTok{read_excel}\NormalTok{(infile) }\OperatorTok\StringTok{ }
\StringTok{                  }\NormalTok{janitor}\OperatorTok{::}\KeywordTok{clean_names}\NormalTok{()}
\end{Highlighting}
\end{Shaded}

We can use skimr to take a quick look at the data to see if there is
anything missing. Unfortunately the output of skimr doesn't play well
with tex.

\subsection{Data clean-up}\label{data-clean-up}

Fix the classes for some columns. For instance, the number of term and
the political party should be a factor. Also, get rid of some messy
columns we won't use.

\begin{Shaded}
\begin{Highlighting}[]
\NormalTok{mps_attributes }\OperatorTok\StringTok{ }\KeywordTok{select}\NormalTok{(}\OperatorTok{-}\NormalTok{notes)}

\NormalTok{term_levels <-}\KeywordTok{c}\NormalTok{(}\StringTok{"First"}\NormalTok{, }\StringTok{"Second"}\NormalTok{, }\StringTok{"Third"}\NormalTok{, }\StringTok{"Fourth"}\NormalTok{,}\StringTok{"Fifth"}\NormalTok{,}
                \StringTok{"Sixth"}\NormalTok{, }\StringTok{"Seventh"}\NormalTok{, }\StringTok{"Eighth"}\NormalTok{, }\StringTok{"Ninth"}\NormalTok{)}
\NormalTok{mps_attributes}\OperatorTok{$}\NormalTok{no_of_term }\OperatorTok\StringTok{ }\KeywordTok{factor}\NormalTok{(term_levels)}

\NormalTok{mps_attributes}\OperatorTok{$}\NormalTok{political_party }\OperatorTok\StringTok{ }\NormalTok{factor}
\end{Highlighting}
\end{Shaded}

The really tricky bit is this - we now have to join this set of
attributes with the data on attendance using name and/or constituency.
There are so many ways in which this can go wrong - variation on
spelling, using or not using titles etc. We will do the best we can! For
starters, make everything lowercase, get rid of space and hope for the
best.

\begin{Shaded}
\begin{Highlighting}[]
\NormalTok{clean_up_strings <-}\ControlFlowTok{function}\NormalTok{(str)\{}
\NormalTok{    str }\OperatorTok\StringTok{ }
\StringTok{    }\NormalTok{tolower }\OperatorTok\StringTok{ }
\StringTok{    }\KeywordTok{gsub}\NormalTok{(}\StringTok{' '}\NormalTok{,}\StringTok{''}\NormalTok{, .)}
\NormalTok{\}}

\NormalTok{mps_attributes}\OperatorTok{$}\NormalTok{mp_name }\OperatorTok\StringTok{ }\NormalTok{clean_up_strings}
\NormalTok{all_sessions}\OperatorTok{$}\NormalTok{mp_name }\OperatorTok\StringTok{ }\NormalTok{clean_up_strings}

\NormalTok{mps_attributes}\OperatorTok{$}\NormalTok{constituency }\OperatorTok\StringTok{ }\NormalTok{clean_up_strings}
\NormalTok{all_sessions}\OperatorTok{$}\NormalTok{constituency   }\OperatorTok\StringTok{ }\NormalTok{clean_up_strings}
\end{Highlighting}
\end{Shaded}

Also, manually fix up some spelling differences between the two files.

\begin{Shaded}
\begin{Highlighting}[]
\NormalTok{## constituency as spelt in mp_attributes = as in all_sessions}
\NormalTok{mps_attributes}\OperatorTok{$}\NormalTok{constituency }\OperatorTok
\StringTok{ }\NormalTok{forcats}\OperatorTok{::}\KeywordTok{fct_recode}\NormalTok{(}\DataTypeTok{ramanathapuram =} \StringTok{"ramanthapuram"}\NormalTok{,}
                     \DataTypeTok{mahesana =} \StringTok{"mehesana"}\NormalTok{,}
                     \DataTypeTok{kushinagar =} \StringTok{"khushinagar"}\NormalTok{,}
                     \DataTypeTok{krishnagiri =} \StringTok{"krisnagiri"}\NormalTok{,}
                     \DataTypeTok{bangaon =} \StringTok{"bongaon"}\NormalTok{)}

\KeywordTok{levels}\NormalTok{(mps_attributes}\OperatorTok{$}\NormalTok{constituency) }\OperatorTok\StringTok{ }\KeywordTok{c}\NormalTok{(}\StringTok{"nominatedanglo-indian"}\NormalTok{)}

\NormalTok{## set this to nominatedanglo-indian}
\NormalTok{mps_attributes[mps_attributes}\OperatorTok{$}\NormalTok{nature_of_membership }\OperatorTok{==}\StringTok{ "Nominated"}\NormalTok{,}
               \StringTok{"constituency"}\NormalTok{] <-}\StringTok{  "nominatedanglo-indian"}
\end{Highlighting}
\end{Shaded}

Join time!

\begin{Shaded}
\begin{Highlighting}[]
\NormalTok{attendance_with_attributes <-}\StringTok{ }\KeywordTok{left_join}\NormalTok{(all_sessions, mps_attributes,}
                                        \DataTypeTok{by =} \StringTok{"constituency"}\NormalTok{)}
\end{Highlighting}
\end{Shaded}

\begin{verbatim}
## Warning: Column `constituency` joining character vector and factor,
## coercing into character vector
\end{verbatim}

\begin{Shaded}
\begin{Highlighting}[]
\NormalTok{## }
\end{Highlighting}
\end{Shaded}

No complaints from R. Check data using skimr and a few rows.


\end{document}
